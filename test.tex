\documentclass[11pt]{article}
\usepackage{palatino}

\usepackage{etex}

\usepackage[english]{babel}
\usepackage[utf8]{inputenc}
\usepackage{fancyhdr}

\usepackage{amsmath,dsfont}
\usepackage{graphicx}%used for the figures
\usepackage{caption}%to caption side by side figures
\usepackage{subcaption}%to caption side by side figures
\usepackage{amssymb}
\usepackage{amsthm}
\usepackage{authblk}
\usepackage{dcpic}
\usepackage[hang,flushmargin]{footmisc}
\usepackage[left=1in,right=1in,top=1in,bottom=1in]{geometry}
\usepackage[ocgcolorlinks,linkcolor=linkblue,citecolor=linkred,urlcolor=linkred]{hyperref}
\usepackage{url}
% \urldef{\Rouches}\url{https://en.wikipedia.org/wiki/Rouch%C3%A9%27s_theorem}

% \urldef{\Rouches}\hyperref[Wikipedia]{https://en.wikipedia.org/wiki/Rouch%C3%A9%27s_theorem}
% 
\usepackage{cleveref}

\usepackage{MnSymbol}
\usepackage{pictexwd}
\usepackage{tikz-cd}
\usepackage{verbatim}
\usepackage{xcolor}
\usepackage{natbib}
\usepackage{bibentry}
\usepackage[normalem]{ulem}



% \usepackage{mathtools}
% \mathtoolsset{showonlyrefs}

\usepackage{mdframed}

%%%% BEGIN added by JH %%%%
%% 

%% show the todo list in the end of the file
\usepackage{todo}
\makeatletter
\renewcommand*\@tododisplay[0]
\makeatother
\let\TodoOriginal\Todo
\renewcommand\Todo[2][To~do]{%
  {\color{red} \TodoOriginal[#1]{\normalsize #2}}%
  \@todotrue%
}


%\usepackage{JH-abbrev}

\usepackage{enumitem}
\usepackage{empheq}

\usepackage{pgf}

\makeatletter
\newcommand\pgfinvisible{\pgfsys@begininvisible}
\newcommand\pgfshown{\pgfsys@endinvisible}
\makeatother

% after JH-abbrev, hide all the 2 pi i factors (show / hide)
\def\CIpidisplay{hide}

\usepackage{showkeys} % show references
\usepackage{seqsplit} % couper les mots
\usepackage{etoolbox}

\usepackage{calc} % pour faire des opérations mathématiques

% nécessite etoolbox
\newcommand{\fullrlap}[1]{%
  \rlap{\kern\dimexpr-\@totalleftmargin+\textwidth+\marginparsep\relax#1}}

% pour que les étiquettes restent dans la marge
% nécessite seqsplit
\renewcommand*\showkeyslabelformat[1]{%
  \hspace{-4pt}\fbox{\parbox[t]{\marginparwidth-5pt}{\raggedright\scriptsize\ttfamily\seqsplit{#1}}}}

% #1 = optional tag name
% #2 = command for future uses
% #3 = equation content
% #4 = equation reference
\newcommand{\restatableeq}[2]{
  #2
  \gdef#1{#2}
}

% tag for subequations
\newcommand{\subtag}[1]{
  \makeatletter
  \def\@currentlabel{#1}
  \makeatother
  \renewcommand{\theequation}{#1\arabic{equation}}
}

\newcommand{\red}[1]{
  {\color{red}{#1}}
}

\newcommand{\CI}{
  \mathrm{CI}
}

\usepackage{environ}
\NewEnviron{killcontents}{}

%%%% END added by JH %%%%


\numberwithin{equation}{section}

\definecolor{linkred}{rgb}{0.75,0,0}
\definecolor{linkblue}{rgb}{0,0,0.75}

\theoremstyle{plain}
\newtheorem{maintheorem}{Theorem}

\theoremstyle{plain}
\newtheorem{theorem}{Theorem}[section]
\newtheorem{lemma}[theorem]{Lemma}
\newtheorem{proposition}[theorem]{Proposition}
\newtheorem{corollary}[theorem]{Corollary}
\newtheorem{conjecture}[theorem]{Conjecture}
\newtheorem{claim}[theorem]{Claim}
\newtheorem{question}[theorem]{Question}


\theoremstyle{definition}
\newtheorem{definition}[theorem]{Definition}
\newtheorem{rem}[theorem]{Remark}
\newtheorem{example}[theorem]{Example}

% \newcommand{\s}{\sigma}
\newcommand{\B}{\overline{B}}
\newcommand{\rout}[1]{\red{\sout{#1}}}
\newcommand{\blue}[1]{\textcolor{blue}{#1}}
\definecolor{kellygreen}{rgb}{0.3, 0.73, 0.09}
\newcommand{\green}[1]{\textcolor{kellygreen}{#1}}
\newcommand{\orange}[1]{\textcolor{orange}{#1}}
\newcommand{\oout}[1]{\orange{\sout{#1}}}


\setlength{\parindent}{0pt}
\setlength{\parskip}{6pt}
\linespread{1.1}


\def\bull{\vrule height 1ex width .8ex depth -.2ex}
\renewcommand{\labelitemi}{\bull}


\pagestyle{plain}
% \fancyhf{}
% \rhead{Project Description \thepage}
% \lhead{Axel Iv\'an S\'aenz Rodr\'iguez}
% \rfoot{}


\begin{document}

\title{\vspace{-8ex} The quantum Heisenberg-Ising spin-1/2 chain on a 1-dimensional ring}

\author{Nik E, Muhammad Faks, and Axel I.~Saenz}
% \author{Axel Iv\'an S\'aenz Rodr\'iguez}
% \affil{University of Virginia}


\date{}

\maketitle

\begin{abstract}
  We numerically solve the quantum Heisenberg-Ising spin-1/2 chain, aka.~the XXZ model, on a 1-dimensional Ring. The numerical solution is powered by the Bethe ansatz. In particular, via the Bethe ansatz, we find a numerical solution by (numerically) solving a system of algebraic equation and checking an initial condition identity. We developed an open-access Python program to implement the numerical solution and numerically verify the necessary initial condition. Additionally, we used the numerical solution to compute some statistics of the XXZ model on a ring, such as the 1-point function and the gap probability. This project is a stepping stone in developing numerical evidence to determine if the XXZ model in the Kardar-Parisi-Zhang (KPZ) universality class, a conjecture put forward in \cite{STW22}.
\end{abstract}




% \tableofcontents

% \begin{center}
%   \textbf{Project Description}
% \end{center}

\section{Introduction}

\subsection{The Model}

The quantum Heisenberg-Ising spin-1/2 chain is a finite dimensional closed quantum system. The evolution of the system is determined by the Schrodinger equation
\begin{equation}
    i\frac{d}{dt} | \Psi (t) \rangle = H  | \Psi (t) \rangle
\end{equation}
where $| \Psi (t) \rangle$ is the wave function and $H$ is Hermitean matrix.

\subsection{Main result}

\begin{theorem}
    \begin{equation}
    | \Psi (t) \rangle = \sum_{x} \left(\sum_{z} \frac{1}{\det(I + M)} \sum_{\sigma \in S_N} A_{\sigma(z)} \prod_{i=1}^{N} z_{\sigma(i)}^{x_i -y_{\sigma(i)-1}} e^{-it \epsilon(z_i)}\right) | x \rangle
    \end{equation}
\end{theorem}







%\bibliographystyle{alpha}
%\bibliography{RingXXZ}

    

\end{document}
