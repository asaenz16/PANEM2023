%\documentclass{amsart}
\documentclass{article}

\usepackage{fullpage}
\usepackage{wrapfig}
\usepackage{amsmath,amsthm}
\usepackage{amssymb}
\usepackage{latexsym}
\usepackage{graphicx,epstopdf}
\usepackage{enumerate}
\thispagestyle{empty}

\usepackage{hyperref}
\hypersetup{
    colorlinks=true,
    linkcolor=blue,
    filecolor=magenta,      
    urlcolor=blue,
    pdftitle={Overleaf Example},
    pdfpagemode=FullScreen,
    }

\urlstyle{same}

%%%%%%%%%%%%%%%%%%%%%%%%%%%%%%%%%%%%%%%%%%%%
%\usepackage[T1]{fontenc}
%\usepackage[sc]{mathpazo}
%\linespread{1.05}         % Palatino needs more leading (space between lines)
%%%%%%%%%%%%%%%%%%%%%%%%%%%%%%%%%%%%%%%%%%%%


\def\rbb{\mathbb{R}}

\def\f{\varphi}
\def\ep{\epsilon}

\newcommand{\E}[1]{\mathbb{E}\left[#1\right]}
\renewcommand{\P}[1]{\mathbb{P}\left\{#1\right\}}

\begin{document}

\begin{center}
\begin{Large}
Advanced Calculus I
\end{Large}

\begin{large}
MTH 311, Fall 2021
\end{large}
\end{center}

\bigskip

\noindent \textbf{Instructor:} Axel Ivan Saenz Rodriguez \hfill \textbf{Office:} KIDD 292\\
\noindent \textbf{E-mail:} \texttt{saenzroa$@$oregonstate.edu} \hfill \textbf{Office Hours:} on Zoom\\
\noindent \textbf{Lectures:} MWF 9:00 - 9:50 am BEXL 103  \hfill  W 2:30-4:30 \\
%\noindent Webpage: \texttt{http://randommath.net} \hfill Wed 11 am - noon\\
\vspace{1mm}\\
\noindent \textbf{GTA:} Benjamin Toomey \hfill \textbf{Office:} KIDD 069\\
\noindent \textbf{E-mail:} \texttt{toomeyb$@$oregonstate.edu} \hfill \textbf{Office Hours:} on Zoom \\
\noindent \textbf{Recitation:} Th 9:00-9:50 am \& 10:00-10:50 am HOV 100 \hfill T 1:00-3:00 pm\\

\noindent \textbf{Mask Requirement.} Masks that cover from your nose to your chin are required for all students in the classroom at all times (unless prior approval of exemption from Disability Access Services is obtained). Masks help protect you, your peers, and me from obtaining a serious and potentially deadly disease. If you do not have a mask, you can obtain one from the Science Succes Center in Kidder Hall, the Library, MU, and Student Health Services. Please note that face shields do not qualify as masks. You are encouraged to review OSU’s policy and ensure you understand the expectations around supporting a safe transition back to an in-person campus. For general information, see Beaver Healthy quick reference guide on CANVAS and visit \href{https://covid.oregonstate.edu}{https://covid.oregonstate.edu}. See COVID contingency plan in a later section for more information on what to do if you have symptoms.

\bigskip

\begin{large}
\noindent \textbf{Course Information}
\end{large}

\bigskip

\noindent \textbf{Prerequisite}:
A minimum grade of C- is required in (MTH 254 or MTH 254H) and MTH 355.

\bigskip

\noindent \textbf{Textbook}

\href{https://www.springer.com/us/book/9781493927111?gclid=Cj0KCQiAr8bwBRD4ARIsAHa4YyKFsQG7Lh8btaedOrsSl1kO4wzCDyS36owaXKVTbMPXB7gwZdOc0r4aAnShEALw_wcB}{Understanding Analysis, Second Edition}
by Abbott, Stephen\\
The class will follow the textbook closely and will cover first five chapters. The textbook is required and is available electronically to student with valid ONID accounts through the Valley library, click \href{https://search.library.oregonstate.edu/primo-explore/fulldisplay?docid=CP71228341450001451&vid=OSU&search_scope=everything&tab=default_tab&lang=en_US&context=L}{here}.

\bigskip

\noindent \textbf{Course Description}
Rigorous development of calculus, axiomatic properties of the real numbers, topology of the real line, convergence of sequences and series of real numbers, functions, limits of functions, basic properties of continuity and derivatives. Brief treatment of Riemann integration, improper integrals, sequences of functions, pointwise and uniform convergence, introductory aspects of multivariable calculus. All courses used to satisfy MTH prerequisites must be completed with C- or better.

\bigskip

\noindent \textbf{Learning Outcomes}
A student completing MTH 311 is expected to understand and create, i.e.~read and write, precise and rigorous mathematical statements, based on the foundations of calculus. This course is often the first introduction to proof writing for many students. As such, emphasis is placed on effective communication of abstract mathematical ideas. The successful student will be a able to read and write concise and easy to follow fundamental proofs regarding the analysis of continuous and differential real functions on the discrete or continuous line.

%\bigskip

%\begin{center}
%\begin{tabular}{|l|l|}
%  	\hline
%  	\multicolumn{2}{|c|}{\textbf{Course Material}} \\
%  	\hline
%	Chapter 1 & Overview and Descriptive Statistics \\
%	Chapter 2 & Probability \\
%	Chapter 3 & Discrete Random Variables and Probability Distributions \\
%	Chapter 4 & Continuous Random Variables and Probability Distributions \\
%	Chapter 5 & Joint Probability Distributions \\
%	Chapter 6 & Statistics and Sampling Distributions \\
%	\hline
%\end{tabular}
%\end{center}
%There may be supplemental information provided from time to time.

\bigskip

\noindent \textbf{Evaluation.}
The grade for the course is based on weekly homeworks, a midterm and a final exam. The final grade is curved, and typically it works out close to this:
\[
\text{A:} \,\, [90-100], \text{B:} \,\, [80-90], \text{C:} \,\, [70-80], \text{D:} \,\, [60-70], \text{F:} \,\, [< 60]
\]
The weight for each component of the grade is given as follows:
\begin{enumerate}
\item \textbf{Homeworks} (50\%). There will be weekly homework assignments, based on the lectures from the previous week and the corresponding content from the textbook. Unless otherwise stated, the homework will be \textbf{due by the beginning of class on Fridays}. Moreover, the assignments will be posted and \textbf{submitted electronically on the CANVAS} site for the course; physical submissions are strongly discouraged.  Students will be evaluated on the accuracy of their work and, also, on the presentation of their work. While students are encouraged to work through problem sets with each other, every students is expected to complete the writing of each assignment individually. The student is expected to neatly write easy to follow and direct arguments for their assignments. Additionally, the student is encouraged to typeset their assignments using LaTeX; see \href{https://www.overleaf.com/learn/latex/Tutorials}{Overleaf} for an introduction and tutorial.
\item \textbf{Midterm} (20\% each). There will be one in-class mideterm exam, covering the topics from the first 5 weeks of classes. The exam will be held on \textbf{Friday, November 5}.
\item \textbf{Final Exam} (30\%). The final exam will be cumulative, covering the topics from the entire course. The exam will be held at \textbf{2pm} on \textbf{Thursday, December 9}.
\end{enumerate}
%There will be two in-class midterms, which will account for 20\% of your grade each, the final exam will be comprehensive and will account for 30\% of your grade. There will be weekly problem sets assigned generally on Wednesdays and due AT THE BEGINNING OF CLASS the following Wednesday. Homework will be worth 30\% of your grade.  %The remaining 10\% will be based on participation in lecture, discussion, and weekly quizzes, which will be cumulative. 


\bigskip

\noindent \textbf{Make-up Policy.}  There will be no makeup given for the homeworks, the midterm exam or the final exam without prior request, approval and arrangement with Professor Saenz Rodriguez. Exceptions will be handled case by case and will require written documentation of a serious reason to miss an exam (e.g., a doctor's note).

\bigskip

\noindent \textbf{Announcements.} All official announcements will be posted on the CANVAS site for the course. Any modifications or additions to the syllabus will be announced on CANVAS. Any and all Zoom links for the course will be posted on CANVAS. Students are responsible to check the CANVAS site periodically to be up to date with the course.

\bigskip

\noindent \textbf{Office Hours.} There are a total of four office hours for the course each week, two by the instructor and two by the GTA. Office hours will be held virtually by Zoom and the links will be available on CANVAS.

\bigskip

\begin{large}
\noindent \textbf{Student Reseources}
\end{large}

\bigskip

\noindent \textbf{Academic Calendar.} All students are subject to the registration and refund deadlines as stated in the Academic Calendar: \href{ https://registrar.oregonstate.edu/osu-academic-calendar}{https://registrar.oregonstate.edu/osu-academic-calendar}.

\bigskip

\noindent \textbf{Statement Regarding Students with Disabilities} Accommodations for students with disabilities are determined and approved by Disability Access Services (DAS). If you, as a student, believe you are eligible for accommodations but have not obtained approval please contact DAS immediately at 541-737-4098 or at \href{ http://ds.oregonstate.edu}{http://ds.oregonstate.edu}.

\bigskip

\noindent \textbf{Student Conduct Expectations} Students are expected to be familiar with Oregon State University’s Statement of Expectations for Student Conduct. Please review and consult material at the following site: \href{https://beav.es/codeofconduct}{https://beav.es/codeofconduct}.

\bigskip

\noindent \textbf{Reach Out for Success} University students encounter setbacks from time to time. If you encounter difficulties and need assistance, it’s important to reach out. Consider discussing the situation with an instructor or academic advisor. Learn about resources that assist with wellness and academic success at \href{https://oregonstate.edu/ReachOut}{https://oregonstate.edu/ReachOut}. If you are in immediate crisis, please contact the Crisis Text Line by texting OREGON to 741-741 or call the National Suicide Prevention Lifeline at 1-800-273-TALK (8255).

\bigskip

\noindent \textbf{Math Resources} The Math and Statistics Learning Center (MSLC) offers personalized tutoring for OSU students with Graduate Teaching Assistants, faculty members, and skilled undergraduate tutors. See \href{https://math.oregonstate.edu/mslc}{https://math.oregonstate.edu/mslc} for operations and hours.

\bigskip

\noindent \textbf{Communication} Questions regarding the homework problems or the textbook are best address during office hours and/or recitation section. If you are unable to make it to office hours you may email your questions to me or setup an appointment by email. For logistical question about the course, please contact me through email. Expect responses to emails to be within normal working hours, 9 am to 5 pm, and allow 24 hours for a response.

\bigskip

\begin{large}
\noindent \textbf{COVID contingency plan}
\end{large}

\bigskip

\noindent \textbf{If you have symptoms of COVID.} Please notify me as soon as possible by email and don't attend class. Additionally, please notify OSU and get tested for COVID, see Quarantine and Isolation Guidelines available on CANVAS. The Office of the Dean of Students can also assist you if you are navigating a range of extenuating life circumstances including but not limited to prolonged illness, hospitalization, financial concerns, etc. They can be reached via Zoom chat or audio Monday through Friday from 9 am to 5 pm, see contact details at \href{https://studentlife.oregonstate.edu/student-info}{https://studentlife.oregonstate.edu/student-info}.

\bigskip

\noindent \textbf{If I am ill.} The class will continue and take place virtually over Zoom. An announcement will be made on CANVAS, potentially right before the beginning of class. Thus, if the instructor is not present by the beginning of class, please check the CANVAS site. The link for the Zoom meeting will be available on CANVAS.

\bigskip

\noindent \textbf{Disclaimer.} Due to uncertainties around COVID, the instructor reserves the right to make changes to the syllabus throughout the terms. Any changes will be announced on CANVAS with as much time as possible.




\end{document}
