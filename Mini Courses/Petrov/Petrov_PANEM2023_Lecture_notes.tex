\documentclass[letterpaper,11pt,oneside,reqno]{article}

%%%%%%%%%%%%%%%%%%%%%%%%%%%%%%%%%%%%%%%%%%%%%%%%%%%%%%%%%%%%

\usepackage[pdftex,backref=page,colorlinks=true,linkcolor=blue,citecolor=red]{hyperref}
\usepackage[alphabetic,nobysame]{amsrefs}

%%%%%%%%%%%%%%%%%%%%%%%%%%%%%%%%%%%%%%%%%%%%%%%%%%%%%%%%%%%%
%main packages
\usepackage{amsmath,amssymb,amsthm,amsfonts,mathtools}
\usepackage{graphicx,color}
\usepackage{upgreek}
\usepackage[mathscr]{euscript}

%equations
\allowdisplaybreaks
\numberwithin{equation}{section}
%tikz
\usepackage{tikz}

%conveniences
\usepackage{array}
\usepackage{adjustbox}
\usepackage{cleveref}
\usepackage{enumerate}
\usepackage{datetime}

%paper geometry
\usepackage[DIV=12]{typearea}

%%%%%%%%%%%%%%%%%%%%%%%%%%%%%%%%%%%%%%%%%%%%%%%%%%%%%%%%%%%%
%draft-specific
\synctex=1
% \usepackage{refcheck,comment}

%%%%%%%%%%%%%%%%%%%%%%%%%%%%%%%%%%%%%%%%%%%%%%%%%%%%%%%%%%%%
%this paper specific
\newcommand{\ssp}{\hspace{1pt}}

%%%%%%%%%%%%%%%%%%%%%%%%%%%%%%%%%%%%%%%%%%%%%%%%%%%%%%%%%%%%
\newtheorem{proposition}{Proposition}[section]
\newtheorem{lemma}[proposition]{Lemma}
\newtheorem{corollary}[proposition]{Corollary}
\newtheorem{theorem}[proposition]{Theorem}
%%%%%%%%%%%%%%%%%%%%%%%%%%%%%%%%%%%%%%%%%%%%%%%%%%%%%%%%%%%%
\theoremstyle{definition}
\newtheorem{definition}[proposition]{Definition}
\newtheorem{remark}[proposition]{Remark}
\newtheorem{example}[proposition]{Example}
%%%%%%%%%%%%%%%%%%%%%%%%%%%%%%%%%%%%%%%%%%%%%%%%%%%%%%%%%%%%
\theoremstyle{remark}
\newtheorem{exercise}{Exercise}[section]

\begin{document}
\title{How to Solve the Stochastic Six Vertex Model}

% OTHER AUTHORS 

% \author{Matthew Nicoletti and Leonid Petrov}
\author{Leonid Petrov}

\date{At \currenttime{}, on \today}

\setcounter{tocdepth}{4}

\maketitle


\section*{Introduction}

These are lecture notes for \href{https://www.math.tamu.edu/conferences/functional_analysis/PANEM.html}{\texttt{PANEM-2023}} at Texas A{}\&M on the integrability and asymptotics of the stochastic six vertex model.



\newpage
\section{Six vertex model through different lenses}
\label{sec:6v_model_lecture}

In the first lecture, we describe the stochastic six vertex model
from two diverse perspectives --- as a model of statistical mechanics,
and as a stochastic particle system.

\subsection{Gibbs measures and the six vertex model}
\label{sub:gibbs_6v}

\subsubsection{Finite-volume Gibbs measures}

We begin with describing the useful framework of \emph{Gibbs measures}.
For simplicity, we work on the two-dimensional lattice $\mathbb{Z}^2$.
Let $\Lambda\subset\mathbb{Z}^{2}$ be a finite subset (for example, a rectangle).
We are interested in \emph{spin configurations} inside $\Lambda$
which are encoded as $\omega=\{\sigma_e\colon e \textnormal{ is an edge in $\Lambda$}\}$,
where $\sigma_e\in\left\{ 0,1 \right\}$.
By an ``edge in $\Lambda$'' we mean that both endpoints of this edge must be inside $\Lambda$.
Each spin configuration is equipped with boundary conditions,
which are fixed spins on all the boundary edges of $\Lambda$
(an edge is called boundary if it connects $\Lambda$ to $\mathbb{Z}^{2}\setminus \Lambda$).

With each spin configuration $\omega$, we associate 
its energy
$H(\omega)\in \mathbb{R}$.
This energy
may depend on global parameters
(e.g., inverse temperature)
and local parameters (e.g., edge capacities or vertex rapidities).
If a particular spin configuration $\omega$ is forbidden, we have $H(\omega)=+\infty$.

\begin{definition}
	\label{def:Gibbs_measure_finite}
	A (finite-volume) \emph{Gibbs measure}
	in $\Lambda$ 
	with fixed boundary conditions 
	and the energy function $H(\cdot)$
	is the probability distribution on spin configurations 
	whose probability weights have the form
	\begin{equation*}
		\mathop{\mathrm{{Prob}}}(\omega)=\frac{1}{Z}\ssp\exp\left\{ -H(\omega) \right\}.
	\end{equation*}
	Here $Z$ is the \emph{partition function}, which is simply the probability normalizing constant.
\end{definition}

\begin{example}[Domino tilings on the square grid]
	\label{example:dominos}
	A \emph{perfect matching} on $\Lambda$ is any subset $M$ of its edges 
	such that every vertex is covered by exactly one 
	edge from $M$.
	For example, here is a perfect matching on the four by four rectangle:
	\begin{equation*}
		\includegraphics[width=.2\textwidth]{./images/domino_4_4.png}
	\end{equation*}
	If the set of allowed spin configurations is the 
	set of perfect matchings (with $\sigma_e=1$ if the edge is included in the matching, and 
	$\sigma_e=0$ otherwise),
	and 
	\begin{equation*}
		H(\omega)=\begin{cases}
			0,&\textnormal{$\omega$ is a perfect matching};\\
			+\infty,&\textnormal{$\omega$ is not a perfect matching},
		\end{cases}
	\end{equation*}
	then the corresponding Gibbs measure
	is the 
	uniform distribution
	on the space of \emph{domino tilings}.
	That is, we identify each covered edge with a $2\times 1$ domino.
	The domino tiling corresponding to the above perfect matching is
	\begin{equation*}
		\includegraphics[width=.2\textwidth]{./images/tiling_4_4.png}
	\end{equation*}
\end{example}

\begin{exercise}
	Compute the number of domino tilings
	of the thin rectangles 
	\begin{enumerate}[(a)]
		\item $2\times n$;
		\item $3\times (2n)$.
	\end{enumerate}
\end{exercise}

Computing partition functions of various Gibbs measures
may be challenging. For example, the number of 
domino tilings of the $8\times 8$ chessboard is 12,988,816,
but its theoretical computation (not via a computer program)
requires several nontrivial steps \cite{Kasteleyn1961}, \cite{temperley1961dimer}. 

Parameter-dependent partition functions represent many important 
quantities across all of mathematics, including various
families of symmetric functions (such as Schur or Hall-Littlewood functions), and related objects.

\subsubsection{Infinite-volume Gibbs measures}

Besides Gibbs measures on configurations on a
finite space as in \Cref{def:Gibbs_measure_finite} with fixed boundary
conditions (``\emph{boxed distributions}''), we are 
interested
in \emph{infinite-volume} Gibbs measures. 

\begin{definition}
	\label{def:infinite_Gibbs}
	A probability measure on spin configurations on 
	an infinite subset $\Lambda_\infty\subseteq\mathbb{Z}^{2}$
	(we will mainly consider the whole plane and the quarter plane $\mathbb{Z}_{\ge0}^{2}$)
	is called (infinite-volume) Gibbs if for any finite $\Lambda\subset \Lambda_\infty$,
	the configuration inside $\Lambda$
	conditioned on the configuration in $\Lambda_\infty\setminus \Lambda$
	is a finite-volume Gibbs measure in the sense of \Cref{def:Gibbs_measure_finite}
	(with boundary conditions imposed by the outside configuration in $\Lambda_\infty\setminus \Lambda$).
\end{definition}

Out of all possible infinite-volume Gibbs measures, we are interested in measures with 
special properties, such as translation invariant and/or ergodic.
A Gibbs measure on $\mathbb{Z}^{2}$ is called \emph{translation invariant}
if its distribution does not change under arbitrary
space translations. 
A Gibbs measure is called \emph{ergodic} (equivalently, \emph{extremal})
if it cannot be represented as a convex combination of 
two other such measures.
Gibbs measures which are translation invariant
and ergodic (within the class of translation invariant measures)
are called \emph{pure states}.

Classifying pure states for a given energy function $H(\cdot)$ is a very
nontrivial problem, and an 
explicit answer is rarely available. 
For the general six vertex model 
(defined in \Cref{subsub:6v_model} below),
the answer is only conjectural
\cite{reshetikhin2010lectures}.

On the other hand, pure states of the six vertex model 
under a special
\emph{free fermion condition}
(which includes the domino model from \Cref{example:dominos})
admit a very explicit description through
determinantal point processes 
(i.e., all correlation functions
of these measures are diagonal minors
of an explicit function in two variables),
which follows from
\cite{Sheffield2008}, 
\cite{KOS2006}.
One of the goals of these lecture notes is to discuss 
the tools and results one would require to extend this classification
beyond the free fermion case.


\begin{remark}
	\label{rmk:inf_Gibbs}
	Certain families of 
	non translation invariant
	infinite-volume
	Gibbs measures (under the free fermion condition)
	power the
	classification of irreducible representations of
	infinite-dimensional unitary group
	and other classical groups
	\cite{Voiculescu1976},
	\cite{VK82CharactersU},
	\cite{BorodinOlsh2011GT},
	\cite{Petrov2012GT}.
	This subject is closely related to symmetric
	functions arising as partition functions
	of Gibbs measures with 
	varying parameters (rapidities)
	along one of the lattice coordinate
	direction which we discuss in the second lecture (\Cref{sec:integrability}).
	There is also a direct link between these Gibbs measures and totally nonnegative triangular or
	full Toeplitz matrices for characters of the infinite
	symmetric group or, respectively, the
	infinite-dimensional unitary group, see \cite{AESW51},
	\cite{Edrei53}, \cite{Boyer1983}.
\end{remark}

\subsubsection{Six vertex model}
\label{subsub:6v_model}

The most general Gibbs property we consider in these notes is that of the 
\emph{asymmetric six vertex model}. 
The six vertex model 
was introduced by Pauling to model
the residual entropy of ice \cite{pauling1935structure}
(see also 
\cite{Nature_square_ice}
for recent experiments
with square ice between two graphene sheets, and \Cref{rmk:ice} below for an exact connection).
The model
has received a lot of attention since the seminal Bethe Ansatz solutions obtained in the 1960's in
\cite{Lieb1967SixVertex}, \cite{YangYang1966}.
We refer to the book 
\cite{baxter2007exactly} for an introduction, and also to
\cite{reshetikhin2010lectures} for a more recent survey of the 
model.

Under the asymmetric six vertex model, 
the allowed spin configurations on the two-dimensional lattice 
are such that locally around each vertex there can be one of the following six configurations:
\begin{equation*}
	\includegraphics[width=.7\textwidth]{./images/6v_def.png}
\end{equation*}
Viewing the edges with spin $\sigma_e=1$
as parts of up-right paths, we can think of six vertex model
configurations as up-right path configurations on the lattice, where paths are allowed to touch at a
vertex.
We denote the six vertex types by $a_1,a_2,b_1,b_2,c_1,c_2$:
\begin{equation}
	\label{eq:6v_weights_picture}
	\includegraphics[width=.7\textwidth]{./images/6v_def1.png}
\end{equation}
See \Cref{fig:ice}, right,
for an example of a global configuration of up-right paths in a rectangle.

Abusing the notation, we also think of 
$a_1,a_2,b_1,b_2,c_1,c_2\ge0$ as the Gibbs weights $e^{-H}$
assigned to each local vertex. That is, the six vertex model Gibbs measure
in a rectangle $\Lambda$ with given boundary conditions 
(that is, with prescribed spin configurations at all edges connecting
$\Lambda$ to $\mathbb{Z}^{2}\setminus\Lambda$)
has probability weights
\begin{equation*}
	\mathop{\mathrm{Prob}}(\omega)=\frac{
		a_1^{\#\{a_1\text{ vertices}\}}
		a_2^{\#\{a_2\text{ vertices}\}}
		b_1^{\#\{b_1\text{ vertices}\}}
		b_2^{\#\{b_2\text{ vertices}\}}
		c_1^{\#\{c_1\text{ vertices}\}}
		c_2^{\#\{c_2\text{ vertices}\}}
	}{Z(a_1,a_2,b_1,b_2,c_1,c_2)}.
\end{equation*}
Here 
$\#\{a_1\text{ vertices}\}$ is the number of 
vertices of type $a_1$ in the configuration $\omega$, and so on.

\begin{remark}[Connection to square ice]
	\label{rmk:ice}
	In the square ice, the oxygen atoms should 
	form a perfect square grid, and each 
	edge contains a hydrogen atom.
	The hydrogen atom on an edge is connected to one of the adjacent oxygens 
	by the chemical bond, and to another oxygen by a weaker hydrogen
	bond. 
	This allows to distinguish two types of edges, and assign
	``spins'' $0$ and $1$ to them. 
	Since each oxygen must have exactly two hydrogen
	atoms attached to it by chemical bonds, we get 
	six possible local configurations around a vertex. See \Cref{fig:ice}
	for an illustration.

	\begin{figure}[htpb]
		\centering
		\includegraphics[width=.3\textwidth]{./images/ice.png}
		\qquad 
		\includegraphics[width=.265\textwidth]{./images/paths.png}
		\caption{Left: a configuration of the square ice. Right:
		the corresponding configuration of the six vertex model in the square grid.}
		\label{fig:ice}
	\end{figure}
\end{remark}

The quantity
\begin{equation}
	\label{eq:delta_6v_general}
	\Delta\coloneqq \frac{a_1a_2+b_1b_2-c_1c_2}{2\sqrt{a_1a_2b_1b_2}}
\end{equation}
plays an important role in the (mostly conjectural) description
of pure phases of the asymmetric six vertex model.
Depending on $\Delta$, there are several regimes of the model:
\begin{enumerate}[$\bullet$]
	\item Ferroelectric: $\Delta>1$;
	\item Anti-ferroelectric: $\Delta<1$;
	\item Disordered: $-1<\Delta<1$;
	\item Free fermion point: $\Delta=0$.
\end{enumerate}

\begin{example}
	[Free fermion specializations]
	\label{example:free_fermion}
	Specializing the weights $a_1,a_2,b_1,b_2,c_1,c_2$ in two different free fermion ways, 
	we obtain the following particular cases:
	\begin{enumerate}[$\bullet$]
		\item If $a_1=a_2=b_1=b_2=1$ and $c_1=c_2=\sqrt 2$, 
			one can map
			six vertex configurations into domino tilings from \Cref{example:dominos}.
			This map is not a bijection between configurations, but instead 
			it splits a $c$-type vertex into two equivalent 
			local configurations of the domino tilings, while preserving the 
			configuration weights.
			This idea goes back to 
			\cite{elkies1992alternating},
			see also \cite{zinn2000six}, \cite{ferrari2006domino}.
			A multiparameter generalization of the domino tiling model
			coming from the free fermion six vertex model 
			was considered in \cite{ABPW2021free},
			see also \cite{Naprienko2023}.
		\item When $a_2=0$ and $a_1=b_1=b_2=c_1=c_2=1$,
			we forbid the intersection of paths. This model can be bijectively mapped into
			a model of \emph{lozenge tilings} (e.g., see \cite{gorin2021lectures} for the definition), 
			with configurations like
			\begin{equation*}
				\includegraphics[width=.4\textwidth]{./images/lozenge_from_vertex.png}
			\end{equation*}
	\end{enumerate}
\end{example}

\begin{exercise}
	\begin{enumerate}[(a)]
		\item 
	Work out the details of the mapping from the 
	free fermion six vertex model with 
	$a_2=0$ and $a_1=b_1=b_2=c_1=c_2=1$ 
	to lozenge tilings. What happens to the boundary conditions?

\item
	If we set $b_1=c_1=u_xv_y$, where $(x,y)$ are the lattice coordinates of a vertex,
	then six vertex configurations
	start depending on the parameters $u_i,v_j$ (which we assume to be generic complex numbers).
	How do these weights translate into the lozenge tiling picture?
	\end{enumerate}
\end{exercise}

In the next 
\Cref{sub:s6v_and_degenerations} 
we consider another important particular case --- the 
\emph{stochastic six vertex model}, which is no longer free fermion.

\begin{remark}[Alternating sign matrices]
	There are other very interesting specializations
	of the six vertex model 
	which are not stochastic
	nor free fermion. Let us only mention
	that when $a_1=a_2=b_1=b_2=c_1=c_2=1$ (so $\Delta=1/2$, which is in the disordered regime),
	the six vertex Gibbs property becomes uniform
	(on configurations of up-right paths which are allowed to touch at a vertex).
	There is a bijection from six vertex configurations to 
	\emph{alternating sign matrices}. 
	This bijection allowed to compute the number of alternating sign matrices
	from a partition functions of the six vertex model.
	We refer to 
	\cite{kuperberg1996another} and
	\cite{Propp2001} for details.
\end{remark}

\subsection{Stochastic six vertex model and its particle system limits}
\label{sub:s6v_and_degenerations}

\subsubsection{Stochastic six vertex model}

Sampling six vertex configurations 
by Glauber dynamics (which proceeds by random local flips)
can be exponentially slow under some conditions on the parameters
\cite{FahrbachRandall2019}.
Here we consider a case of parameters 
under which the sampling 
(in the case when the top and the right boundaries are ``free'')
can instead be done by running a Markov chain just once.

\begin{definition}
	\label{def:s6v}
	The six vertex model with parameters $a_1,a_2,b_1,b_2,c_1,c_2$ 
	\eqref{eq:6v_weights_picture}
	is called \emph{stochastic}
	if 
	\begin{equation}
		\label{eq:s6v_condition}
		a_1=a_2=1,\qquad 
		b_1+c_1=b_2+c_2=1,\qquad 
		b_1,b_2\in[0,1].
	\end{equation}
	The stochastic six vertex model depends on the 
	two remaining parameters
	$(b_1,b_2)$.
\end{definition}

Let us explain how conditions \eqref{eq:s6v_condition}
simplify the sampling of the model.
Take a finite rectangle $\Lambda\subset\mathbb{Z}^{2}$,
and equip it with 
\emph{free outgoing boundary conditions},
for which the locations of 
outgoing paths on the right and the top boundaries of the rectangle
are not specified. 
Due to the 
stochasticity condition \eqref{eq:s6v_condition},
the partition function of the stochastic six vertex Gibbs measure 
in $\Lambda$ with these boundary conditions is simply equal to $1$.

The stochastic six vertex configuration in $\Lambda$
can be sampled by running a row-to-row Markov chain
based on the vertex weights,
see \Cref{fig:P_u_free} for an illustration.


\begin{figure}[htpb]
	\centering
	\includegraphics[width=\textwidth]{./images/fig_P_u_free.pdf}
	\caption{Sampling the configuration of the stochastic six vertex model
		in a rectangle $\Lambda$ with free outgoing boundary conditions
		by the row-to-row Markov chain. At each step, 
		we perform the sequential update (from left to right)
		in a single row using 
		the incoming paths from the left and from the row below.
		For example, after one step the probability
		of getting the displayed configuration is
		equal to 
		$b_2c_2c_1$.}
	\label{fig:P_u_free}
\end{figure}

\medskip

The stochastic six vertex model was introduced in \cite{GwaSpohn1992}.
The stochastic specialization of the parameters allows to study this particular case of the 
six vertex
model using the toolbox from stochastic interacting particle systems
\cite{Liggett1985}, by treating the vertical direction as time.
In \Cref{subsub:diagonal_limit_ASEP,subsub:PushTASEP_limit}
below we consider two limits 
of the stochastic six vertex model to 
more familiar continuous time interacting particle systems.

\subsubsection{Limit shape}
\label{subsub:limit_shape_BCG}

We are now in a position 






\subsubsection{Diagonal limit}
\label{subsub:diagonal_limit_ASEP}


\subsubsection{Hall-Littlewood PushTASEP limit}
\label{subsub:PushTASEP_limit}







\subsection{Gibbs properties of the stochastic six vertex model}
\label{sub:Gibbs_s6v}

Amol's lemma


\subsection{Basic coupling and colored (multispecies) models}
\label{sub:colored}

??? did not explain in the lecture

\begin{figure}[htpb]
	\centering
	\includegraphics[width=.3\textwidth]{./images/CS6V.png}
	\quad
	\includegraphics[width=.3\textwidth]{./images/S6V.png}
	\quad
	\raisebox{2pt}{\includegraphics[width=.29\textwidth]{./images/stoch6v1.png}}
	\caption{Colored stochastic six vertex model, its monochrome version,
	and a smaller simulation of the monochrome six vertex model.}
	\label{fig:CS6V}
\end{figure}




\subsection{Stationary distributions and hydrodynamics}
\label{sub:hydrodynamic_analysis}

This should be fine, but maybe not too short



Bernoulli is Stationary; also for all the limits we had.

\subsection{Limit shape and fluctuation problem}
\label{sub:limit_shape_problem}










\newpage
\section{Integrability}
\label{sec:integrability}





\newpage
\section{Asymptotics}
\label{sec:asymptotics}







\newpage
\bibliographystyle{alpha}
\bibliography{bib}

\medskip

\textsc{University of Virginia, Charlottesville, VA}

E-mail: \texttt{lenia.petrov@gmail.com}


\end{document}
