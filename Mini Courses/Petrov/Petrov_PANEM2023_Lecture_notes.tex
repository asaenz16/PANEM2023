\documentclass[letterpaper,11pt,oneside,reqno]{article}

%%%%%%%%%%%%%%%%%%%%%%%%%%%%%%%%%%%%%%%%%%%%%%%%%%%%%%%%%%%%

\usepackage[pdftex,backref=page,colorlinks=true,linkcolor=blue,citecolor=red]{hyperref}
\usepackage[alphabetic,nobysame]{amsrefs}

%%%%%%%%%%%%%%%%%%%%%%%%%%%%%%%%%%%%%%%%%%%%%%%%%%%%%%%%%%%%
%main packages
\usepackage{amsmath,amssymb,amsthm,amsfonts,mathtools}
\usepackage{graphicx,color}
\usepackage{upgreek}
\usepackage[mathscr]{euscript}

%equations
\allowdisplaybreaks
\numberwithin{equation}{section}
%tikz
\usepackage{tikz}

%conveniences
\usepackage{array}
\usepackage{adjustbox}
\usepackage{cleveref}
\usepackage{enumerate}
\usepackage{datetime}

%paper geometry
\usepackage[DIV=12]{typearea}

%%%%%%%%%%%%%%%%%%%%%%%%%%%%%%%%%%%%%%%%%%%%%%%%%%%%%%%%%%%%
%draft-specific
\synctex=1
% \usepackage{refcheck,comment}

%%%%%%%%%%%%%%%%%%%%%%%%%%%%%%%%%%%%%%%%%%%%%%%%%%%%%%%%%%%%
%this paper specific
\newcommand{\ssp}{\hspace{1pt}}

%%%%%%%%%%%%%%%%%%%%%%%%%%%%%%%%%%%%%%%%%%%%%%%%%%%%%%%%%%%%
\newtheorem{proposition}{Proposition}[section]
\newtheorem{lemma}[proposition]{Lemma}
\newtheorem{corollary}[proposition]{Corollary}
\newtheorem{theorem}[proposition]{Theorem}
%%%%%%%%%%%%%%%%%%%%%%%%%%%%%%%%%%%%%%%%%%%%%%%%%%%%%%%%%%%%
\theoremstyle{definition}
\newtheorem{definition}[proposition]{Definition}
\newtheorem{remark}[proposition]{Remark}
\newtheorem{example}[proposition]{Example}
%%%%%%%%%%%%%%%%%%%%%%%%%%%%%%%%%%%%%%%%%%%%%%%%%%%%%%%%%%%%
\theoremstyle{remark}
\newtheorem{exercise}{Exercise}[section]

\begin{document}
\title{How to Solve the Stochastic Six Vertex Model}

% OTHER AUTHORS 

% \author{Matthew Nicoletti and Leonid Petrov}
\author{Leonid Petrov}

\date{At \currenttime{}, on \today}

\setcounter{tocdepth}{4}

\maketitle


\section*{Introduction}

These are lecture notes for \href{https://www.math.tamu.edu/conferences/functional_analysis/PANEM.html}{\texttt{PANEM-2023}} at Texas A{}\&M on the integrability and asymptotics of the stochastic six vertex model.



\newpage
\section{Six vertex model through different lenses}
\label{sec:6v_model_lecture}

In the first lecture, we describe the stochastic six vertex model
from two diverse perspectives --- as a model of statistical mechanics,
and as a stochastic particle system.

\subsection{Gibbs measures and the six vertex model}
\label{sub:gibbs_6v}

\subsubsection{Finite-volume Gibbs measures}

We begin with describing the useful framework of \emph{Gibbs measures}.
For simplicity, we work on the two-dimensional lattice $\mathbb{Z}^2$.
Let $\Lambda\subset\mathbb{Z}^{2}$ be a finite subset (for example, a rectangle).
We are interested in \emph{spin configurations} inside $\Lambda$
which are encoded as $\omega=\{\sigma_e\colon e \textnormal{ is an edge in $\Lambda$}\}$,
where $\sigma_e\in\left\{ 0,1 \right\}$.
By an ``edge in $\Lambda$'' we mean that both endpoints of this edge must be inside $\Lambda$.
Each spin configuration is equipped with boundary conditions,
which are fixed spins on all the boundary edges of $\Lambda$
(an edge is called boundary if it connects $\Lambda$ to $\mathbb{Z}^{2}\setminus \Lambda$).

With each spin configuration $\omega$, we associate 
its energy
$H(\omega)\in \mathbb{R}$.
This energy
may depend on global parameters
(e.g., inverse temperature)
and local parameters (e.g., edge capacities or vertex rapidities).
If a particular spin configuration $\omega$ is forbidden, we have $H(\omega)=+\infty$.

\begin{definition}
	\label{def:Gibbs_measure_finite}
	A (finite-volume) \emph{Gibbs measure}
	in $\Lambda$ 
	with fixed boundary conditions 
	and the energy function $H(\cdot)$
	is the probability distribution on spin configurations 
	whose probability weights have the form
	\begin{equation*}
		\mathop{\mathrm{{Prob}}}(\omega)=\frac{1}{Z}\ssp\exp\left\{ -H(\omega) \right\}.
	\end{equation*}
	Here $Z$ is the \emph{partition function}, which is simply the probability normalizing constant.
\end{definition}

\begin{example}[Domino tilings on the square grid]
	\label{ex:dominos}
	A \emph{perfect matching} on $\Lambda$ is any subset $M$ of its edges 
	such that every vertex is covered by exactly one 
	edge from $M$.
	For example, here is a perfect matching on the four by four rectangle:
	\begin{equation*}
		\includegraphics[width=.2\textwidth]{./images/domino_4_4.png}
	\end{equation*}
	If the set of allowed spin configurations is the 
	set of perfect matchings,
	and 
	\begin{equation*}
		H(\omega)=\begin{cases}
			0,&\textnormal{$\omega$ is a perfect matching};\\
			+\infty,&\textnormal{$\omega$ is not a perfect matching},
		\end{cases}
	\end{equation*}
	then the corresponding Gibbs measure
	is the 
	uniform distribution
	on the space of \emph{domino tilings}.
	That is, we identify each covered edge with a $2\times 1$ domino.
	The domino tiling corresponding to the above perfect matching is
	\begin{equation*}
		\includegraphics[width=.2\textwidth]{./images/tiling_4_4.png}
	\end{equation*}
\end{example}

\begin{exercise}
	Compute the number of domino tilings
	of the thin rectangles 
	\begin{enumerate}[(a)]
		\item $2\times n$;
		\item $3\times (2n)$.
	\end{enumerate}
\end{exercise}

Computing partition functions of various Gibbs measures
may be challenging. For example, the number of 
domino tilings of the $8\times 8$ chessboard is 12,988,816,
but its theoretical computation (not via a computer program)
requires several nontrivial steps \cite{Kasteleyn1961}, \cite{temperley1961dimer}. 

Parameter-dependent partition functions represent many important 
quantities across all of mathematics, including various
families of symmetric functions (such as Schur or Hall-Littlewood functions), and related objects.

\subsubsection{Infinite-volume Gibbs measures}

Besides Gibbs measures on configurations on a
finite space as in \Cref{def:Gibbs_measure_finite} with fixed boundary
conditions (``\emph{boxed distributions}''), we are 
interested
in \emph{infinite-volume} Gibbs measures. 

\begin{definition}
	\label{def:infinite_Gibbs}
	A probability measure on spin configurations on 
	an infinite subset $\Lambda_\infty\subseteq\mathbb{Z}^{2}$
	(we will mainly consider the whole plane and the quarter plane $\mathbb{Z}_{\ge0}^{2}$)
	is called (infinite-volume) Gibbs if for any finite $\Lambda\subset \Lambda_\infty$,
	the configuration inside $\Lambda$
	conditioned on the configuration in $\Lambda_\infty\setminus \Lambda$
	is a finite-volume Gibbs measure in the sense of \Cref{def:Gibbs_measure_finite}
	(with boundary conditions imposed by the outside configuration in $\Lambda_\infty\setminus \Lambda$).
\end{definition}

Out of all possible infinite-volume Gibbs measures, we are interested in measures with 
special properties, such as translation invariant and/or ergodic.
A Gibbs measure on $\mathbb{Z}^{2}$ is called \emph{translation invariant}
if its distribution does not change under arbitrary
space translations. 
A Gibbs measure is called \emph{ergodic} (equivalently, \emph{extremal})
if it cannot be represented as a convex combination of 
two other such measures.
Gibbs measures which are translation invariant
and ergodic (within the class of translation invariant measures)
are called \emph{pure states}.

Classifying pure states for a given energy function $H(\cdot)$ is a very
nontrivial problem, and an 
explicit answer is rarely available. 
For the general six vertex model 
(defined in \Cref{subsub:6v_model} below),
the answer is only conjectural
\cite{reshetikhin2010lectures}.

On the other hand, pure states of the six vertex model 
under a special
\emph{free fermion condition}
(which includes the domino model from \Cref{ex:dominos})
admit a very explicit description through
determinantal point processes 
(i.e., all correlation functions
of these measures are diagonal minors
of an explicit function in two variables),
which follows from
\cite{Sheffield2008}, 
\cite{KOS2006}.
One of the goals of these lecture notes is to discuss 
the tools and results one would require to extend this classification
beyond the free fermion case.


\begin{remark}
	\label{rmk:inf_Gibbs}
	Certain families of 
	non translation invariant
	infinite-volume
	Gibbs measures (under the free fermion condition)
	power the
	classification of irreducible representations of
	infinite-dimensional unitary group
	and other classical groups
	\cite{Voiculescu1976},
	\cite{VK82CharactersU},
	\cite{BorodinOlsh2011GT},
	\cite{Petrov2012GT}.
	This subject is closely related to symmetric
	functions arising as partition functions
	of Gibbs measures with 
	varying parameters (rapidities)
	along one of the lattice coordinate
	direction which we discuss in the second lecture (\Cref{sec:integrability}).
	There is also a direct link between these Gibbs measures and totally nonnegative triangular or
	full Toeplitz matrices for characters of the infinite
	symmetric group or, respectively, the
	infinite-dimensional unitary group, see \cite{AESW51},
	\cite{Edrei53}, \cite{Boyer1983}.
\end{remark}

\subsubsection{Six vertex model}
\label{subsub:6v_model}

The most general Gibbs property we consider in these notes is that of the 
\emph{asymmetric six vertex model}. 
The six vertex model 
was introduced by physicists to model
the residual entropy of ice \cite{pauling1935structure},
and has received a lot of attention since
\cite{Lieb1967SixVertex}, \cite{YangYang1966}.
We refer to the book 
\cite{baxter2007exactly} for an introduction, and also to
\cite{reshetikhin2010lectures} for a more recent survey of the 
model.

Under the asymmetric six vertex model, 

\colorbox{yellow}{\parbox{.7\textwidth}{def; pictures; what spins mean; what $e^{-H}$ means for the Gibbs
property; conditioning on outside configuration means conditioning on the boundary edge occupations}}


Specializing the weights $a_1,a_2,b_1,b_2,c_1,c_2$, we obtain the following particular cases:
\begin{enumerate}[$\bullet$]
	\item 
\end{enumerate}
This list is not exclusive, and in the next 
\Cref{sub:s6v_and_degenerations} 
we consider another important particular case --- the 
\emph{stochastic six vertex model}.

\subsection{Stochastic six vertex model and its particle system limits}
\label{sub:s6v_and_degenerations}



\subsection{Gibbs properties of the stochastic six vertex model}
\label{sub:Gibbs_s6v}


\subsection{Basic coupling and colored (multispecies) models}
\label{sub:colored}

\begin{figure}[htpb]
	\centering
	\includegraphics[width=.3\textwidth]{./images/CS6V.png}
	\quad
	\includegraphics[width=.3\textwidth]{./images/S6V.png}
	\caption{Colored stochastic six vertex model and its monochrome version.}
	\label{fig:CS6V}
\end{figure}




\subsection{Stationary distributions and hydrodynamics}
\label{sub:hydrodynamic_analysis}




Bernoulli is Stationary; also for all the limits we had.

\subsection{Limit shape and fluctuation problem}
\label{sub:limit_shape_problem}










\newpage
\section{Integrability}
\label{sec:integrability}





\newpage
\section{Asymptotics}
\label{sec:asymptotics}







\newpage
\bibliographystyle{alpha}
\bibliography{bib}

\medskip

\textsc{University of Virginia, Charlottesville, VA}

E-mail: \texttt{lenia.petrov@gmail.com}


\end{document}
